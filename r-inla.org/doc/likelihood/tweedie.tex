\documentclass[a4paper,11pt]{article}
\usepackage[scale={0.8,0.9},centering,includeheadfoot]{geometry}
\usepackage{amstext}
\usepackage{amsmath}
\usepackage{verbatim}

\begin{document}
\section*{Tweedie}

\subsection*{Parametrisation}

The Tweedie distribution\footnote{This documentation follows the
    notation in \emph{ Likelihood-based and Bayesian methods for
        Tweedie compound Poisson linear mixed models, by Yanwei Zhang,
        Stat Comput (2013) 23:743–757, DOI 10.1007/s11222-012-9343-7}}
is a compond Poisson-Gamma model, where
\begin{displaymath}
    Y = \sum_{i=1}^{N} X_i,
\end{displaymath}
and $\{X_i\}$ are iid Gamma variables with parameter
$(\alpha, \gamma)$ so that the mean of $X_i$ is $\alpha\gamma$ and
variance $\alpha\gamma^{2}$, and $N$ is (independent) Poisson with
mean $\lambda$. Since $N$ can be $0$ with a positive probability, then
the Tweedie distribution have a singleton in zero and is continuous
for $y>0$.

We will use the following reparametersation
\begin{displaymath}
    \mu = \lambda \alpha \gamma,\qquad
    p = \frac{\alpha+2}{\alpha+1},\qquad
    \frac{\phi}{w} = \frac{\lambda^{1-p} (\alpha\gamma)^{2-p}}{2-p}
\end{displaymath}
where $w>0$ is a fixed scaling, so the mean of $Y$ is $\mu>0$,
variance is $\frac{\phi}{w} \mu^{p}$ where $1<p<2$, and $\phi$ is a
dispersion parameter.

\subsection*{Link-function}

The linkfunction is given as 
\begin{displaymath}
    \log(\mu) =  \eta
\end{displaymath}
where $\eta$  is the linear predictor.

\subsection*{Hyperparameters}

The hyperparameters are ${\theta}=(\theta_1,\theta_2)$, where
\begin{displaymath}
    p=1 + \frac{\exp(\theta_1)}{1+\exp(\theta_1)}, \qquad 1<p<2.
\end{displaymath}
and
\begin{displaymath}
    \phi=\exp(\theta_2), \qquad \phi>0
\end{displaymath}
The priors are given on $(\theta_1,\theta_2)$.
\subsection*{Specification}

\begin{itemize}
\item $\text{family}=\texttt{tweedie}$
\item Required arguments: $y$ (and optional $w$ through option
    \texttt{scale})
\end{itemize}

\subsubsection*{Hyperparameter spesification and default values}
%% DO NOT EDIT!
%% This file is generated automatically from models.R
\begin{description}
	\item[doc] Tweedie distribution
	\item[hyper]\ 
	 \begin{description}
	 	\item[theta1]\ 
	 	 \begin{description}
	 	 	\item[hyperid] 102101
	 	 	\item[name] p
	 	 	\item[short.name] p
	 	 	\item[initial] 0
	 	 	\item[fixed] FALSE
	 	 	\item[prior] normal
	 	 	\item[param] 0 10
	 	 	\item[to.theta] \verb!function(x, interval = c(1.0, 2.0)) log(-(interval[1] - x) / (interval[2] - x))!
	 	 	\item[from.theta] \verb!function(x, interval = c(1.0, 2.0)) interval[1] + (interval[2] - interval[1]) * exp(x) / (1.0 + exp(x))!
	 	 \end{description}
	 	\item[theta2]\ 
	 	 \begin{description}
	 	 	\item[hyperid] 102201
	 	 	\item[name] dispersion
	 	 	\item[short.name] phi
	 	 	\item[initial] 0
	 	 	\item[fixed] FALSE
	 	 	\item[prior] loggamma
	 	 	\item[param] 1 0.1
	 	 	\item[to.theta] \verb!function(x) log(x)!
	 	 	\item[from.theta] \verb!function(x) exp(x)!
	 	 \end{description}
	 \end{description}
	\item[status] experimental
	\item[survival] FALSE
	\item[discrete] FALSE
	\item[link] default log
	\item[pdf] tweedie
\end{description}


\subsection*{Example}

In the following example we estimate the parameters in a simulated
example.
\verbatiminput{example-tweedie.R}

\subsection*{Notes}

This distribution is experimental, and changes will occur.

\end{document}

% LocalWords: 

%%% Local Variables: 
%%% TeX-master: t
%%% End: 
