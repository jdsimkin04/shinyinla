\documentclass[a4paper,11pt]{article}
\usepackage[scale={0.8,0.9},centering,includeheadfoot]{geometry}
\usepackage{amstext}
\usepackage{amsmath}
\usepackage{verbatim}

\begin{document}
\section*{The Gammacount-likelihood}

\subsection*{Parametrisation}

The Gammacount-distribution is a discrete probability distribution on
$0, 1, 2, 3, \ldots$,  where
\begin{displaymath}
    \text{Prob}(y) = G(y\alpha, \beta) - G((y+1)\alpha, \beta)
\end{displaymath}
where
\begin{displaymath}
    G(a, b) = \frac{1}{\Gamma(a)} \int_0^{b} x^{a-1}\exp(-x) dx.
\end{displaymath}
The reciprocal of the expected waiting time depends on the linear
predictor
\begin{displaymath}
    (\alpha/\beta)^{-1} = E\exp(\eta),
\end{displaymath}
so that high values of $\eta$ corresponds to high values of $y$, and
low values of $\eta$ corresponds to low values of $y$, and $\log(E)$
is an offset. In the case where $\alpha=1$, then we are back to the
Poisson.

\subsection*{Link-function}

The linear predictor $\eta$ is linked to the reciprocal of the
expected waiting time, by a log-link,
\begin{displaymath}
    (\alpha/\beta)^{-1} = E\exp(\eta), \qquad E >0.
\end{displaymath}

\subsection*{Hyperparameter}

The hyperparameter is the parameter $\alpha$, which is
represented as
\begin{displaymath}
    \alpha = \exp(\theta)
\end{displaymath}
and the prior is defined on $\theta$.

\subsection*{Specification}

\begin{itemize}
\item $\text{family}=\texttt{gammacount}$
\item Required arguments: $y$ (and $E$, default one)
\end{itemize}

\subsubsection*{Hyperparameter spesification and default values}
%% DO NOT EDIT!
%% This file is generated automatically from models.R
\begin{description}
	\item[doc] A Gamma generalisation of the Poisson likelihood
	\item[hyper]\ 
	 \begin{description}
	 	\item[theta]\ 
	 	 \begin{description}
	 	 	\item[hyperid] 59001
	 	 	\item[name] log alpha
	 	 	\item[short.name] alpha
	 	 	\item[initial] 0
	 	 	\item[fixed] FALSE
	 	 	\item[prior] pc.gammacount
	 	 	\item[param] 3
	 	 	\item[to.theta] \verb!function(x) log(x)!
	 	 	\item[from.theta] \verb!function(x) exp(x)!
	 	 \end{description}
	 \end{description}
	\item[survival] FALSE
	\item[discrete] FALSE
	\item[link] default log
	\item[status] experimental
	\item[pdf] gammacount
\end{description}



\subsection*{Example}

In the following example we estimate the parameters in a simulated
example.
\verbatiminput{example-gammacount.R}

\subsection*{Notes}

None.

\end{document}


% LocalWords: 

%%% Local Variables: 
%%% TeX-master: t
%%% End: 
