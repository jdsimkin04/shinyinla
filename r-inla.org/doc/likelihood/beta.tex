\documentclass[a4paper,11pt]{article}
\usepackage[scale={0.8,0.9},centering,includeheadfoot]{geometry}
\usepackage{amstext}
\usepackage{amsmath}
\usepackage{verbatim}

\begin{document}
\section*{The Beta-distribution}

\subsection*{Parametrisation}

The Beta-distribution has the following density
\begin{displaymath}
    \pi(y) = \frac{1}{B(a, b)} y^{a-1}(1-y)^{b-1}, \qquad 0 < y < 1,
    \quad a>0, \quad b > 0
\end{displaymath}
where $B(a, b)$
is the Beta-function 
\begin{displaymath}
    B(a, b) = \frac{\Gamma(a)\Gamma(b)}{\Gamma(a+b)}
\end{displaymath}
and $\Gamma(x)$ is the Gamma-function.
The (re-)parameterisation used is
\begin{displaymath}
    \mu = \frac{a}{a+b}, \qquad 0 < \mu < 1
\end{displaymath}
and
\begin{displaymath}
    \phi = a + b, \qquad \phi > 0,
\end{displaymath}
as it makes
\begin{displaymath}
    \text{E}(y) = \mu \quad\text{and}\quad \text{Var}(y) = \frac{\mu(1-\mu)}{1+\phi}.
\end{displaymath}
The parameter $\phi$ is know as the \emph{precision parameter}, since
for fixed $\mu$, the larger $\phi$ the smaller the variance of $y$.
The parameters $\{a, b\}$ are given as $\{\mu, \phi\}$ as follows,
\begin{displaymath}
    a = \mu \phi \quad\text{and}\quad  b = -\mu \phi + \phi.
\end{displaymath}

In some applications then observations close to 0 or 1, are censored
and represented as exactly 0 and 1. For this, we introduced a censor
value $0 < \delta<1/2$ and treat all $y \le \delta$ or
$y \ge 1-\delta$ as censored observations. By default, no censoring is
applied ($\delta=0$).


\subsection*{Link-function}

The linear predictor $\eta$ is linked to the mean $\mu$ using a
default logit-link
\begin{displaymath}
    \mu = \frac{\exp(\eta)}{1+\exp(\eta)}.
\end{displaymath}

\subsection*{Hyperparameter}

The hyperparameter is the precision parameter $\phi$, which is
represented as
\begin{displaymath}
    \phi = s_i \exp(\theta)
\end{displaymath}
where $s = (s_i) > 0$ is a fixed scaling, and the prior is defined on
$\theta$.

\subsection*{Specification}

\begin{itemize}
\item $\text{family}=\texttt{beta}$
\item Required argument: $y$
\item Optional argument: $s$ (argument \texttt{scale}, default all 1, $s>0$)
\item Optional argument: truncation limit $0 \le \delta < 1/2$
    (argument \texttt{beta.trunctation}, $\delta=0$ means no trunctation).
\end{itemize}

\subsubsection*{Hyperparameter spesification and default values}
\documentclass[a4paper,11pt]{article}
\usepackage[scale={0.8,0.9},centering,includeheadfoot]{geometry}
\usepackage{amstext}
\usepackage{amsmath}
\usepackage{verbatim}

\begin{document}
\section*{The Beta-distribution}

\subsection*{Parametrisation}

The Beta-distribution has the following density
\begin{displaymath}
    \pi(y) = \frac{1}{B(a, b)} y^{a-1}(1-y)^{b-1}, \qquad 0 < y < 1,
    \quad a>0, \quad b > 0
\end{displaymath}
where $B(a, b)$
is the Beta-function 
\begin{displaymath}
    B(a, b) = \frac{\Gamma(a)\Gamma(b)}{\Gamma(a+b)}
\end{displaymath}
and $\Gamma(x)$ is the Gamma-function.
The (re-)parameterisation used is
\begin{displaymath}
    \mu = \frac{a}{a+b}, \qquad 0 < \mu < 1
\end{displaymath}
and
\begin{displaymath}
    \phi = a + b, \qquad \phi > 0,
\end{displaymath}
as it makes
\begin{displaymath}
    \text{E}(y) = \mu \quad\text{and}\quad \text{Var}(y) = \frac{\mu(1-\mu)}{1+\phi}.
\end{displaymath}
The parameter $\phi$ is know as the \emph{precision parameter}, since
for fixed $\mu$, the larger $\phi$ the smaller the variance of $y$.
The parameters $\{a, b\}$ are given as $\{\mu, \phi\}$ as follows,
\begin{displaymath}
    a = \mu \phi \quad\text{and}\quad  b = -\mu \phi + \phi.
\end{displaymath}

In some applications then observations close to 0 or 1, are censored
and represented as exactly 0 and 1. For this, we introduced a censor
value $0 < \delta<1/2$ and treat all $y \le \delta$ or
$y \ge 1-\delta$ as censored observations. By default, no censoring is
applied ($\delta=0$).


\subsection*{Link-function}

The linear predictor $\eta$ is linked to the mean $\mu$ using a
default logit-link
\begin{displaymath}
    \mu = \frac{\exp(\eta)}{1+\exp(\eta)}.
\end{displaymath}

\subsection*{Hyperparameter}

The hyperparameter is the precision parameter $\phi$, which is
represented as
\begin{displaymath}
    \phi = s_i \exp(\theta)
\end{displaymath}
where $s = (s_i) > 0$ is a fixed scaling, and the prior is defined on
$\theta$.

\subsection*{Specification}

\begin{itemize}
\item $\text{family}=\texttt{beta}$
\item Required argument: $y$
\item Optional argument: $s$ (argument \texttt{scale}, default all 1, $s>0$)
\item Optional argument: truncation limit $0 \le \delta < 1/2$
    (argument \texttt{beta.trunctation}, $\delta=0$ means no trunctation).
\end{itemize}

\subsubsection*{Hyperparameter spesification and default values}
\documentclass[a4paper,11pt]{article}
\usepackage[scale={0.8,0.9},centering,includeheadfoot]{geometry}
\usepackage{amstext}
\usepackage{amsmath}
\usepackage{verbatim}

\begin{document}
\section*{The Beta-distribution}

\subsection*{Parametrisation}

The Beta-distribution has the following density
\begin{displaymath}
    \pi(y) = \frac{1}{B(a, b)} y^{a-1}(1-y)^{b-1}, \qquad 0 < y < 1,
    \quad a>0, \quad b > 0
\end{displaymath}
where $B(a, b)$
is the Beta-function 
\begin{displaymath}
    B(a, b) = \frac{\Gamma(a)\Gamma(b)}{\Gamma(a+b)}
\end{displaymath}
and $\Gamma(x)$ is the Gamma-function.
The (re-)parameterisation used is
\begin{displaymath}
    \mu = \frac{a}{a+b}, \qquad 0 < \mu < 1
\end{displaymath}
and
\begin{displaymath}
    \phi = a + b, \qquad \phi > 0,
\end{displaymath}
as it makes
\begin{displaymath}
    \text{E}(y) = \mu \quad\text{and}\quad \text{Var}(y) = \frac{\mu(1-\mu)}{1+\phi}.
\end{displaymath}
The parameter $\phi$ is know as the \emph{precision parameter}, since
for fixed $\mu$, the larger $\phi$ the smaller the variance of $y$.
The parameters $\{a, b\}$ are given as $\{\mu, \phi\}$ as follows,
\begin{displaymath}
    a = \mu \phi \quad\text{and}\quad  b = -\mu \phi + \phi.
\end{displaymath}

In some applications then observations close to 0 or 1, are censored
and represented as exactly 0 and 1. For this, we introduced a censor
value $0 < \delta<1/2$ and treat all $y \le \delta$ or
$y \ge 1-\delta$ as censored observations. By default, no censoring is
applied ($\delta=0$).


\subsection*{Link-function}

The linear predictor $\eta$ is linked to the mean $\mu$ using a
default logit-link
\begin{displaymath}
    \mu = \frac{\exp(\eta)}{1+\exp(\eta)}.
\end{displaymath}

\subsection*{Hyperparameter}

The hyperparameter is the precision parameter $\phi$, which is
represented as
\begin{displaymath}
    \phi = s_i \exp(\theta)
\end{displaymath}
where $s = (s_i) > 0$ is a fixed scaling, and the prior is defined on
$\theta$.

\subsection*{Specification}

\begin{itemize}
\item $\text{family}=\texttt{beta}$
\item Required argument: $y$
\item Optional argument: $s$ (argument \texttt{scale}, default all 1, $s>0$)
\item Optional argument: truncation limit $0 \le \delta < 1/2$
    (argument \texttt{beta.trunctation}, $\delta=0$ means no trunctation).
\end{itemize}

\subsubsection*{Hyperparameter spesification and default values}
\documentclass[a4paper,11pt]{article}
\usepackage[scale={0.8,0.9},centering,includeheadfoot]{geometry}
\usepackage{amstext}
\usepackage{amsmath}
\usepackage{verbatim}

\begin{document}
\section*{The Beta-distribution}

\subsection*{Parametrisation}

The Beta-distribution has the following density
\begin{displaymath}
    \pi(y) = \frac{1}{B(a, b)} y^{a-1}(1-y)^{b-1}, \qquad 0 < y < 1,
    \quad a>0, \quad b > 0
\end{displaymath}
where $B(a, b)$
is the Beta-function 
\begin{displaymath}
    B(a, b) = \frac{\Gamma(a)\Gamma(b)}{\Gamma(a+b)}
\end{displaymath}
and $\Gamma(x)$ is the Gamma-function.
The (re-)parameterisation used is
\begin{displaymath}
    \mu = \frac{a}{a+b}, \qquad 0 < \mu < 1
\end{displaymath}
and
\begin{displaymath}
    \phi = a + b, \qquad \phi > 0,
\end{displaymath}
as it makes
\begin{displaymath}
    \text{E}(y) = \mu \quad\text{and}\quad \text{Var}(y) = \frac{\mu(1-\mu)}{1+\phi}.
\end{displaymath}
The parameter $\phi$ is know as the \emph{precision parameter}, since
for fixed $\mu$, the larger $\phi$ the smaller the variance of $y$.
The parameters $\{a, b\}$ are given as $\{\mu, \phi\}$ as follows,
\begin{displaymath}
    a = \mu \phi \quad\text{and}\quad  b = -\mu \phi + \phi.
\end{displaymath}

In some applications then observations close to 0 or 1, are censored
and represented as exactly 0 and 1. For this, we introduced a censor
value $0 < \delta<1/2$ and treat all $y \le \delta$ or
$y \ge 1-\delta$ as censored observations. By default, no censoring is
applied ($\delta=0$).


\subsection*{Link-function}

The linear predictor $\eta$ is linked to the mean $\mu$ using a
default logit-link
\begin{displaymath}
    \mu = \frac{\exp(\eta)}{1+\exp(\eta)}.
\end{displaymath}

\subsection*{Hyperparameter}

The hyperparameter is the precision parameter $\phi$, which is
represented as
\begin{displaymath}
    \phi = s_i \exp(\theta)
\end{displaymath}
where $s = (s_i) > 0$ is a fixed scaling, and the prior is defined on
$\theta$.

\subsection*{Specification}

\begin{itemize}
\item $\text{family}=\texttt{beta}$
\item Required argument: $y$
\item Optional argument: $s$ (argument \texttt{scale}, default all 1, $s>0$)
\item Optional argument: truncation limit $0 \le \delta < 1/2$
    (argument \texttt{beta.trunctation}, $\delta=0$ means no trunctation).
\end{itemize}

\subsubsection*{Hyperparameter spesification and default values}
\input{../hyper/likelihood/beta.tex}


\subsection*{Example}

In the following example we estimate the parameters in a simulated
example.
\verbatiminput{example-beta.R}

In this example we add truncation.
\verbatiminput{example-beta-trunc.R}

\subsection*{Notes}

None.

\end{document}


% LocalWords:  hyperparameter overdispersion Hyperparameters nbinomial

%%% Local Variables: 
%%% TeX-master: t
%%% End: 



\subsection*{Example}

In the following example we estimate the parameters in a simulated
example.
\verbatiminput{example-beta.R}

In this example we add truncation.
\verbatiminput{example-beta-trunc.R}

\subsection*{Notes}

None.

\end{document}


% LocalWords:  hyperparameter overdispersion Hyperparameters nbinomial

%%% Local Variables: 
%%% TeX-master: t
%%% End: 



\subsection*{Example}

In the following example we estimate the parameters in a simulated
example.
\verbatiminput{example-beta.R}

In this example we add truncation.
\verbatiminput{example-beta-trunc.R}

\subsection*{Notes}

None.

\end{document}


% LocalWords:  hyperparameter overdispersion Hyperparameters nbinomial

%%% Local Variables: 
%%% TeX-master: t
%%% End: 



\subsection*{Example}

In the following example we estimate the parameters in a simulated
example.
\verbatiminput{example-beta.R}

In this example we add truncation.
\verbatiminput{example-beta-trunc.R}

\subsection*{Notes}

None.

\end{document}


% LocalWords:  hyperparameter overdispersion Hyperparameters nbinomial

%%% Local Variables: 
%%% TeX-master: t
%%% End: 
