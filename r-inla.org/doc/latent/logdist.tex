\documentclass[a4paper,11pt]{article}
\usepackage[scale={0.8,0.9},centering,includeheadfoot]{geometry}
\usepackage{amstext}
\usepackage{listings}
\begin{document}

\section*{logdist effect of a covariate}

\subsection*{Parametrization}

This model implements a non-linear effect of a positive covariate $x$
as a part of the linear predictor,
\begin{displaymath}
    \beta \left(1 + \exp(\alpha_1\log(x) - \alpha_2 x\right)
\end{displaymath}
where $\beta\in\Re$, $\alpha_1, \alpha_2 \in \Re^{+}$ and $x\ge 0$.

\subsection*{Hyperparameters}

This model has three hyperparameters, the scaling $\beta$, $\alpha_1$
and $\alpha_2$. 
\begin{displaymath}
    \theta_1 = \beta \qquad \theta_2 = \log(\alpha_1) \qquad  \theta_3 = \log(\alpha_2)
\end{displaymath}
and the priors are given for $\theta_1, \theta_2$ and $\theta_3$.


\subsection*{Specification}

\begin{verbatim}
    f(x, model="logdist", hyper = ..., precision = <precision>)
\end{verbatim}
where \texttt{precision} is the precision for the tiny noise used to
implement this as a latent model. 

\subsubsection*{Hyperparameter specification and default values}
%% DO NOT EDIT!
%% This file is generated automatically from models.R
\begin{description}
	\item[doc] A nonlinear model of a covariate
	\item[hyper]\ 
	 \begin{description}
	 	\item[theta1]\ 
	 	 \begin{description}
	 	 	\item[hyperid] 39021
	 	 	\item[name] beta
	 	 	\item[short.name] b
	 	 	\item[initial] 1
	 	 	\item[fixed] FALSE
	 	 	\item[prior] normal
	 	 	\item[param] 0 1
	 	 	\item[to.theta] \verb!function(x) x!
	 	 	\item[from.theta] \verb!function(x) x!
	 	 \end{description}
	 	\item[theta2]\ 
	 	 \begin{description}
	 	 	\item[hyperid] 39022
	 	 	\item[name] alpha1
	 	 	\item[short.name] a1
	 	 	\item[initial] 0
	 	 	\item[fixed] FALSE
	 	 	\item[prior] loggamma
	 	 	\item[param] 0.1 1
	 	 	\item[to.theta] \verb!function(x) log(x)!
	 	 	\item[from.theta] \verb!function(x) exp(x)!
	 	 \end{description}
	 	\item[theta3]\ 
	 	 \begin{description}
	 	 	\item[hyperid] 39023
	 	 	\item[name] alpha2
	 	 	\item[short.name] a2
	 	 	\item[initial] 0
	 	 	\item[fixed] FALSE
	 	 	\item[prior] loggamma
	 	 	\item[param] 0.1 1
	 	 	\item[to.theta] \verb!function(x) log(x)!
	 	 	\item[from.theta] \verb!function(x) exp(x)!
	 	 \end{description}
	 \end{description}
	\item[constr] FALSE
	\item[nrow.ncol] FALSE
	\item[augmented] FALSE
	\item[aug.factor] 1
	\item[aug.constr] 
	\item[n.div.by] 
	\item[n.required] FALSE
	\item[set.default.values] FALSE
	\item[status] experimental
	\item[pdf] logdist
\end{description}


\subsection*{Example}

\begin{verbatim}
logdist = function(x, beta,  alpha)
{
    return (beta * (1 + exp(alpha[1] * log(x) - alpha[2] * x)))
}

n = 1000
s=0.1
x = runif(n)
beta = 1
alpha = c(1, 0.5)
## start at the true values
hyper = list(
    beta = list(initial = beta), 
    a1 = list(initial = log(alpha[1])),
    a2 = list(initial = log(alpha[2])))
## start somewhere else
hyper = list(
    beta = list(initial = 1), 
    a1 = list(initial = 0), 
    a2 = list(initial = 0))

y = logdist(x, beta,  alpha) + rnorm(n, sd = s)
r = (inla(y ~ -1 + f(x, model="logdist", hyper = hyper), 
          data = data.frame(y, x),
          family = "gaussian",
          verbose=TRUE, 
          control.inla = list(h=0.0001), 
          control.family = list(
              hyper = list(
                  prec = list(
                      initial = log(1/s^2),
                      fixed = TRUE)))))
summary(r)
\end{verbatim}

\subsection*{Notes}
None

\end{document}

% LocalWords: 

%%% Local Variables: 
%%% TeX-master: t
%%% End: 
