\documentclass[a4paper,11pt]{article}
\usepackage[scale={0.8,0.9},centering,includeheadfoot]{geometry}
\usepackage{amstext}
\usepackage{verbatim}
\begin{document}

\section*{Prior for the intercept in the skew-normal link}

\subsection*{Parametrisation}
The skew-normal link parameterise the intercept implicitely through
the quantile level $q$ as a function of the skewness, which for zero
skewness equals the normal intercept in the probit link,
\begin{equation}
    \mu = \Phi^{-1}(q).
\end{equation}
Further the quantile level $q$ is represented by 
\begin{equation}
    q = \frac{\exp(\theta)}{1+\exp(\theta)}
\end{equation}
for the (internal) hyperparameter $\theta$. The
\texttt{linksnintercept} prior is the implied prior for $\theta$ when
$\mu$ is Normal with a given mean and precision.

Note that zero precision is interpreted that the Normal density is
uniform with unit density.

\subsection*{Specification}
\begin{center}
    \texttt{..., prior="linksnintercept", param=c(<mean>, <precision>),...}
\end{center}

\subsection*{Example}

This example just shows that the implied prior for the intercept is
the same for probit and snlink.

\begin{verbatim}
n = 200
Ntrials = 200
x = rnorm(n, sd = 0.5)
eta = x
skew <- 0.0
prob = inla.link.invsn(eta, skew = skew, intercept = 0.75)
y = rbinom(n, size = Ntrials,  prob = prob)

r = inla(y ~ 1 + x,
         family = "binomial",
         data = data.frame(y, x),
         Ntrials = Ntrials,
         control.fixed = list(remove.names = "(Intercept)",
                              prec = 1), 
         control.family = list(
             control.link = list(
                 model = "sn",
                 hyper = list(
                     skew = list(initial = 0,
                                 fixed = TRUE),
                     intercept = list(param = c(0, 1))))))

rr = inla(y ~ 1 + x,
         family = "binomial",
         data = data.frame(y, x),
         Ntrials = Ntrials,
         control.fixed = list(prec = 1, prec.intercept = 1), 
         control.family = list(
             control.link = list(
                 model = "probit")), 
         verbose = TRUE)

r$mlik - rr$mlik
\end{verbatim}


\subsection*{Notes}

This is the default prior for the intercept in the skew-normal link. 

\end{document}


% LocalWords:  hyperparameters param gaussian hyperparameter

%%% Local Variables: 
%%% TeX-master: t
%%% End: 
